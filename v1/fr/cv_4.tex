%%%%%%%%%%%%%%%%%%%%%%%%%%%%%%%%%%%%%%%%%
% Medium Length Professional CV
% LaTeX Template
% Version 2.0 (8/5/13)
%
% This template has been downloaded from:
% http://www.LaTeXTemplates.com
%
% Original author:
% Trey Hunner (http://www.treyhunner.com/)
%
% Important note:
% This template requires the resume.cls file to be in the same directory as the
% .tex file. The resume.cls file provides the resume style used for structuring the
% document.
%
%%%%%%%%%%%%%%%%%%%%%%%%%%%%%%%%%%%%%%%%%

%----------------------------------------------------------------------------------------
%	PACKAGES AND OTHER DOCUMENT CONFIGURATIONS
%----------------------------------------------------------------------------------------

\documentclass{resume} % Use the custom resume.cls style

\usepackage[left=0.75in,top=0.6in,right=0.75in,bottom=0.6in]{geometry} % Document margins
\usepackage[utf8]{inputenc}
\usepackage[T1]{fontenc}
\name{Nicolas Hamilakis} % Your name
\address{nhamilakis.github.io} % Your secondary addess (optional)
\address{(+33)~$\cdot$~06 21 60 20 96 \\ nick.hamilakis562@gmail.com} % Your phone number and email

\begin{document}

%----------------------------------------------------------------------------------------
%	EDUCATION SECTION
%----------------------------------------------------------------------------------------

\begin{rSection}{Formations}
	
\begin{rSubsection}{ Université Paris Diderot}{2014-2018}{}{Paris 7}
	\item Master Informatique (Languages \& Programmation)
\end{rSubsection}
%----------------------------------------------------------------------------------------
\begin{rSubsection}{ Université Pierre \& Marie Curie }{2010-2014}{}{Paris 6}
	\item Licences Sciences et Technologie mention Informatique.
\end{rSubsection}
%----------------------------------------------------------------------------------------
\begin{rSubsection}{ Université Panthéon Sorbonne}{2009-2010}{}{Paris 1}
	\item Cours de civilisation française
\end{rSubsection}

%----------------------------------------------------------------------------------------
\begin{rSubsection}{ 1er Lycée de Ierapetra }{2009}{}{Crete}
	\item Baccalauréat Science \& Technologie
	\item Math,Physique, Informatique et Finance
\end{rSubsection}

\end{rSection}

%----------------------------------------------------------------------------------------
%	WORK EXPERIENCE SECTION
%----------------------------------------------------------------------------------------

\begin{rSection}{Expérience Professionnel}

\begin{rSubsection}{SAS Majoris-Conseil, projet MylocalPhone}{Août 2017 - }{IT Maintenance \& Android development (CDI)}{Paris}
\item Maintenance technique,développement,test et installation de smartphones Android.
\end{rSubsection}

%------------------------------------------------

\begin{rSubsection}{UGC SA, Informatique département R\&D  }{March 2016 - July 2016}{Stage}{Paris}
\item Développement d'une API et d'un site intranet en ASP.NET/C\# et AngularJS
\end{rSubsection}

%------------------------------------------------

\begin{rSubsection}{C.R.O.U.S de Paris}{January 2008 - April 2010}{}{Paris}
\item Concierge et étudiant référent
\end{rSubsection}

%------------------------------------------------

\begin{rSubsection}{KFC France}{2013 - 2014}{Employée}{Paris}
	\item
\end{rSubsection}
\end{rSection}

%----------------------------------------------------------------------------------------
%	TECHNICAL STRENGTHS SECTION
%----------------------------------------------------------------------------------------

\begin{rSection}{Compétences Technique}

\begin{tabular}{ @{} >{\bfseries}l @{\hspace{6ex}} l }
Langue de Programmation & Java, C/C++, OCaml, Scala, Prolog \\
Script \& Tools & Python, Perl, Bash/Linux, SQL (Postgres, MySQL), Git \\
Web & Html,JavaScript (JQuery, AngularJS), CSS, PHP,  LaTex \\
\end{tabular}

\end{rSection}

%----------------------------------------------------------------------------------------
%	EXAMPLE SECTION
%----------------------------------------------------------------------------------------

\begin{rSection}{Projets Informatique}

\begin{rSubsection}{Elysium}{Android/Java - Cordova,html,css,JavaScript(AngularJS)}{Android App }{}
	\item Effectuer plusieurs tâches de maintenance sur une application launcher en Android (Java). Développement d'un plugin cordova pour effectue plusieurs tâche non accessible en Javascript.
\end{rSubsection}


\begin{rSubsection}{File Manager}{C\#/ASP.NET MVC - html,css,JavaScript(AngularJS)}{Intranet Application \& API}{}
	\item Création d'une API et d'un site Intranet pour l'indexage et gestion de fichier sur plusieurs serveurs différents.Gestion de droit, recherche avec plusieurs critères
\end{rSubsection}

\begin{rSubsection}{Linux Distrubution}{C/Makefiles/Linux Configuration}{Linux Kernel Compilation and Virtualisation}{}
	\item Compiler, package and virtualiser(qemu) une version minimale du noyau Linux. 
\end{rSubsection}

\begin{rSubsection}{Hopix}{Ocaml/Yacc}{Compiler}{}
	\item Implémentation d'une extension à un compilateur existant pour incorporer un nouveau langage (appeler Hopix). Utilisation de Menhir un générateur LR(1) de parseur pour Ocaml.
\end{rSubsection}

\begin{rSubsection}{Simulation de voiture}{SCADE}{Scade Synchronous Application}{}
	\item Création d'un conducteur intelligent capable parcourir plusieurs chemins sur un circuit donnée. Gestion de capteur d'information (Capteur de couleur, sonar, etc..).
\end{rSubsection}

\begin{rSubsection}{UML Editor}{C++/Qt}{GUI PC Application}{}
	\item Création d'une application permettant de crée plusieurs schéma UML et de généré du code en différent langage de programmation (C++, Java..).
\end{rSubsection}

\begin{rSubsection}{Lego MindStorm}{Java/Linux}{University project }{}
	\item Construction et programmation d'une voiture intelligente en utilisant le module Lego MindStorm et l'extension java leJOS.
\end{rSubsection}


\begin{rSubsection}{Flood-it}{C/GTK}{Mini Game for PC}{}
	\item Implémentation du jeu d'inondation (Flood-it), et développement de plusieurs algorithme de joueurs intelligent.
\end{rSubsection}

\begin{rSubsection}{Simulation de voiture de Course}{Java}{Algorithm/Problem Solving}{}
	\item Création d'un conducteur intelligent capable de parcourir de façon optimale un circuit de voiture (Djikstra plus court chemin).
\end{rSubsection}

\end{rSection}

%----------------------------------------------------------------------------------------
\begin{rSection}{Divers}

	\begin{tabular}{ @{} >{\bfseries}l @{\hspace{6ex}} l }
		Langue & Grecque | Français | Anglais  \\
	\end{tabular}
	
\end{rSection}
%----------------------------------------------------------------------------------------


\end{document}
