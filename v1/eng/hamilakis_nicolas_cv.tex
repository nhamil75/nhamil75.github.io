%%%%%%%%%%
% File Created: Friday, 31st August 2018 1:38:45 pm
% Author: Hamilakis Nicolas (nick.hamilakis562@gmail.com)
%-----
% Last Modified: Friday, 31st August 2018 1:38:45 pm
% Modified By: Hamilakis Nicolas (nick.hamilakis562@gmail.com>)
%-----
% Copyright © - 2018 Hamilakis Nicolas,  - 
% LOG:
%%%%%%%%%%%%


%%%%%%%%%%%%%%%%%%%%%%%%%%%%%%%%%%%%%%%%%
% Medium Length Professional CV
% LaTeX Template
% Version 2.0 (8/5/13)
%
% This template has been downloaded from:
% http://www.LaTeXTemplates.com
%
% Original author:
% Trey Hunner (http://www.treyhunner.com/)
%
% Important note:
% This template requires the resume.cls file to be in the same directory as the
% .tex file. The resume.cls file provides the resume style used for structuring the
% document.
%
%%%%%%%%%%%%%%%%%%%%%%%%%%%%%%%%%%%%%%%%%

%----------------------------------------------------------------------------------------
%	PACKAGES AND OTHER DOCUMENT CONFIGURATIONS
%----------------------------------------------------------------------------------------

\documentclass{resume} % Use the custom resume.cls style
\usepackage[utf8]{inputenc}
\usepackage[T1]{fontenc}
\usepackage[left=0.75in,top=0.6in,right=0.75in,bottom=0.6in]{geometry} % Document margins

\name{Nicolas Hamilakis} % Your name
\address{nhamilakis.github.io} % Your secondary addess (optional)
\address{(+33)~$\cdot$~06 21 60 20 96 \\ nick.hamilakis562@gmail.com} % Your phone number and email

\begin{document}

%----------------------------------------------------------------------------------------
%	EDUCATION SECTION
%----------------------------------------------------------------------------------------

\begin{rSection}{Education}
	
\begin{rSubsection}{ University Paris Diderot}{2014-2018}{}{Paris 7}
	\item Master in Computer Science (Languages \& Programming)
\end{rSubsection}
%----------------------------------------------------------------------------------------
\begin{rSubsection}{ University Pierre \& Marie Curie }{2010-2014}{}{Paris 6}
	\item Licences Sciences et Technologie mention Informatique.
\end{rSubsection}
%----------------------------------------------------------------------------------------
\begin{rSubsection}{ University Panthéon Sorbonne}{2009-2010}{}{Paris 1}
	\item Cours de civilisation française
\end{rSubsection}

%----------------------------------------------------------------------------------------
\begin{rSubsection}{ 1st High school of Ierapetra }{2009}{}{Crete}
	\item High school diploma in sciences \& technology
	\item Math,Physics, Computer Sciences and Finance
\end{rSubsection}

\end{rSection}

%----------------------------------------------------------------------------------------
%	WORK EXPERIENCE SECTION
%----------------------------------------------------------------------------------------

\begin{rSection}{Experience}

	\begin{rSubsection}{LSCP Ecole Normale Supérieur, project BabyCloud}{April 2018 - September 2018 }{Full Stack Development (Internship)}{Paris}
		\item Development of a mobile multi-platform application (Ionic, AngularJS).
		\item Development of a back-end service (Python, Flask)
		\item Set-up of a back-end database (PostgreSQL)
	\end{rSubsection}

\begin{rSubsection}{SAS Majoris-Conseil, project MylocalPhone}{August 2017 - Present}{IT Maintenance \& Android development}{Paris}
\item Technical maintenance,development,testing and on site installation of Android smartphones in hotels.
\end{rSubsection}

%------------------------------------------------

\begin{rSubsection}{UGC SA, Informatics R\&D department }{March 2016 - July 2016}{Internship}{Paris}
\item Developing an API \& an intranet website in ASP.NET/C\# and AngularJS
\end{rSubsection}

%------------------------------------------------

\begin{rSubsection}{C.R.O.U.S de Paris}{January 2008 - April 2010}{}{Paris}
\item Host in a student residence.
\end{rSubsection}

%------------------------------------------------

\begin{rSubsection}{KFC France}{2013 - 2014}{Employee}{Paris}
	\item
\end{rSubsection}
\end{rSection}

%----------------------------------------------------------------------------------------
%	TECHNICAL STRENGTHS SECTION
%----------------------------------------------------------------------------------------

\begin{rSection}{Technical Skills}

\begin{tabular}{ @{} >{\bfseries}l @{\hspace{6ex}} l }
Computer Languages & Java, C/C++, OCaml, Scala, Prolog \\
Script \& Tools & Python, Perl, Bash/Linux, SQL (Postgres, MySQL), Git \\
Web & Html,JavaScript (JQuery, AngularJS), CSS, PHP,  LaTex \\
\end{tabular}

\end{rSection}

%----------------------------------------------------------------------------------------
%	EXAMPLE SECTION
%----------------------------------------------------------------------------------------

\begin{rSection}{Projects}

\begin{rSubsection}{Elysium}{Android/Java - Cordova,html,css,JavaScript(AngularJS)}{Android App }{}
	\item Performed various maintenance and development tasks in an Android launcher application targeted for custom use in devices rented to hotels. 
\end{rSubsection}


\begin{rSubsection}{File Manager}{C\#/ASP.NET MVC - html,css,JavaScript(AngularJS)}{Intranet Application \& API}{}
	\item Creating an API that indexes and manages a large list of files accross various servers in a Windows Environment. Creating an intranet website capable of searching tagging and managing the file system.
\end{rSubsection}

\begin{rSubsection}{Linux Distrubution}{C/Makefiles/Linux Configuration}{Linux Kernel Compilation and Virtualisation}{}
	\item Compile, package and virtualise(qemu) a minimal version of the linux kernel. 
\end{rSubsection}

\begin{rSubsection}{Hopix}{Ocaml/Yacc}{Compiler}{}
	\item Implementing an Extension to an existing compiler written in Ocaml for a made up language called Hopix. Using Menhir an LR(1) Ocaml parser generator.
\end{rSubsection}

\begin{rSubsection}{Driving Simulation}{SCADE}{Scade Synchronous Application}{}
	\item Creating a driver that can follow an itinerary within a Scade implementation (Synchronous Real-Time Programming).
\end{rSubsection}

\begin{rSubsection}{UML Editor}{C++/Qt}{GUI PC Application}{}
	\item Creating a gui interface to build UML diagrams \& implementing a code generator.
\end{rSubsection}

\begin{rSubsection}{Lego MindStorm}{Java/Linux}{University project }{}
	\item Building and programming a lego mindstorm project to participate in a competition. Follow a line on the ground, locate an object and collect it. Using a java implementation of the lego OS.
\end{rSubsection}


\begin{rSubsection}{Flood-it}{C/GTK}{Mini Game for PC}{}
	\item Implementation of the Flood-it game in C. Creating \& implementing various solving algorithms. 	
\end{rSubsection}

\begin{rSubsection}{Racing Cars Simulation}{Java}{Algorithm/Problem Solving}{}
	\item Creating a smart driver that can complete a circuit in a given format. Elements of algorithmic (Djikstra shortest path algorithm \& finding the most optimized path). 
\end{rSubsection}

\end{rSection}

%----------------------------------------------------------------------------------------
\begin{rSection}{Various}

	\begin{tabular}{ @{} >{\bfseries}l @{\hspace{6ex}} l }
		Languages & Greek | French | English  \\
	\end{tabular}
	
\end{rSection}
%----------------------------------------------------------------------------------------


\end{document}
